% Options for packages loaded elsewhere
\PassOptionsToPackage{unicode}{hyperref}
\PassOptionsToPackage{hyphens}{url}
\PassOptionsToPackage{dvipsnames,svgnames,x11names}{xcolor}
%
\documentclass[
  letterpaper,
  DIV=11,
  numbers=noendperiod]{scrreprt}

\usepackage{amsmath,amssymb}
\usepackage{iftex}
\ifPDFTeX
  \usepackage[T1]{fontenc}
  \usepackage[utf8]{inputenc}
  \usepackage{textcomp} % provide euro and other symbols
\else % if luatex or xetex
  \usepackage{unicode-math}
  \defaultfontfeatures{Scale=MatchLowercase}
  \defaultfontfeatures[\rmfamily]{Ligatures=TeX,Scale=1}
\fi
\usepackage{lmodern}
\ifPDFTeX\else  
    % xetex/luatex font selection
\fi
% Use upquote if available, for straight quotes in verbatim environments
\IfFileExists{upquote.sty}{\usepackage{upquote}}{}
\IfFileExists{microtype.sty}{% use microtype if available
  \usepackage[]{microtype}
  \UseMicrotypeSet[protrusion]{basicmath} % disable protrusion for tt fonts
}{}
\makeatletter
\@ifundefined{KOMAClassName}{% if non-KOMA class
  \IfFileExists{parskip.sty}{%
    \usepackage{parskip}
  }{% else
    \setlength{\parindent}{0pt}
    \setlength{\parskip}{6pt plus 2pt minus 1pt}}
}{% if KOMA class
  \KOMAoptions{parskip=half}}
\makeatother
\usepackage{xcolor}
\setlength{\emergencystretch}{3em} % prevent overfull lines
\setcounter{secnumdepth}{5}
% Make \paragraph and \subparagraph free-standing
\ifx\paragraph\undefined\else
  \let\oldparagraph\paragraph
  \renewcommand{\paragraph}[1]{\oldparagraph{#1}\mbox{}}
\fi
\ifx\subparagraph\undefined\else
  \let\oldsubparagraph\subparagraph
  \renewcommand{\subparagraph}[1]{\oldsubparagraph{#1}\mbox{}}
\fi


\providecommand{\tightlist}{%
  \setlength{\itemsep}{0pt}\setlength{\parskip}{0pt}}\usepackage{longtable,booktabs,array}
\usepackage{calc} % for calculating minipage widths
% Correct order of tables after \paragraph or \subparagraph
\usepackage{etoolbox}
\makeatletter
\patchcmd\longtable{\par}{\if@noskipsec\mbox{}\fi\par}{}{}
\makeatother
% Allow footnotes in longtable head/foot
\IfFileExists{footnotehyper.sty}{\usepackage{footnotehyper}}{\usepackage{footnote}}
\makesavenoteenv{longtable}
\usepackage{graphicx}
\makeatletter
\def\maxwidth{\ifdim\Gin@nat@width>\linewidth\linewidth\else\Gin@nat@width\fi}
\def\maxheight{\ifdim\Gin@nat@height>\textheight\textheight\else\Gin@nat@height\fi}
\makeatother
% Scale images if necessary, so that they will not overflow the page
% margins by default, and it is still possible to overwrite the defaults
% using explicit options in \includegraphics[width, height, ...]{}
\setkeys{Gin}{width=\maxwidth,height=\maxheight,keepaspectratio}
% Set default figure placement to htbp
\makeatletter
\def\fps@figure{htbp}
\makeatother
% definitions for citeproc citations
\NewDocumentCommand\citeproctext{}{}
\NewDocumentCommand\citeproc{mm}{%
  \begingroup\def\citeproctext{#2}\cite{#1}\endgroup}
\makeatletter
 % allow citations to break across lines
 \let\@cite@ofmt\@firstofone
 % avoid brackets around text for \cite:
 \def\@biblabel#1{}
 \def\@cite#1#2{{#1\if@tempswa , #2\fi}}
\makeatother
\newlength{\cslhangindent}
\setlength{\cslhangindent}{1.5em}
\newlength{\csllabelwidth}
\setlength{\csllabelwidth}{3em}
\newenvironment{CSLReferences}[2] % #1 hanging-indent, #2 entry-spacing
 {\begin{list}{}{%
  \setlength{\itemindent}{0pt}
  \setlength{\leftmargin}{0pt}
  \setlength{\parsep}{0pt}
  % turn on hanging indent if param 1 is 1
  \ifodd #1
   \setlength{\leftmargin}{\cslhangindent}
   \setlength{\itemindent}{-1\cslhangindent}
  \fi
  % set entry spacing
  \setlength{\itemsep}{#2\baselineskip}}}
 {\end{list}}
\usepackage{calc}
\newcommand{\CSLBlock}[1]{\hfill\break\parbox[t]{\linewidth}{\strut\ignorespaces#1\strut}}
\newcommand{\CSLLeftMargin}[1]{\parbox[t]{\csllabelwidth}{\strut#1\strut}}
\newcommand{\CSLRightInline}[1]{\parbox[t]{\linewidth - \csllabelwidth}{\strut#1\strut}}
\newcommand{\CSLIndent}[1]{\hspace{\cslhangindent}#1}

\KOMAoption{captions}{tableheading}
\makeatletter
\@ifpackageloaded{bookmark}{}{\usepackage{bookmark}}
\makeatother
\makeatletter
\@ifpackageloaded{caption}{}{\usepackage{caption}}
\AtBeginDocument{%
\ifdefined\contentsname
  \renewcommand*\contentsname{Table of contents}
\else
  \newcommand\contentsname{Table of contents}
\fi
\ifdefined\listfigurename
  \renewcommand*\listfigurename{List of Figures}
\else
  \newcommand\listfigurename{List of Figures}
\fi
\ifdefined\listtablename
  \renewcommand*\listtablename{List of Tables}
\else
  \newcommand\listtablename{List of Tables}
\fi
\ifdefined\figurename
  \renewcommand*\figurename{Figure}
\else
  \newcommand\figurename{Figure}
\fi
\ifdefined\tablename
  \renewcommand*\tablename{Table}
\else
  \newcommand\tablename{Table}
\fi
}
\@ifpackageloaded{float}{}{\usepackage{float}}
\floatstyle{ruled}
\@ifundefined{c@chapter}{\newfloat{codelisting}{h}{lop}}{\newfloat{codelisting}{h}{lop}[chapter]}
\floatname{codelisting}{Listing}
\newcommand*\listoflistings{\listof{codelisting}{List of Listings}}
\makeatother
\makeatletter
\makeatother
\makeatletter
\@ifpackageloaded{caption}{}{\usepackage{caption}}
\@ifpackageloaded{subcaption}{}{\usepackage{subcaption}}
\makeatother
\ifLuaTeX
  \usepackage{selnolig}  % disable illegal ligatures
\fi
\usepackage{bookmark}

\IfFileExists{xurl.sty}{\usepackage{xurl}}{} % add URL line breaks if available
\urlstyle{same} % disable monospaced font for URLs
\hypersetup{
  pdftitle={Change Management},
  pdfauthor={Witek ten Hove},
  colorlinks=true,
  linkcolor={blue},
  filecolor={Maroon},
  citecolor={Blue},
  urlcolor={Blue},
  pdfcreator={LaTeX via pandoc}}

\title{Change Management}
\usepackage{etoolbox}
\makeatletter
\providecommand{\subtitle}[1]{% add subtitle to \maketitle
  \apptocmd{\@title}{\par {\large #1 \par}}{}{}
}
\makeatother
\subtitle{F-cluster HAN Business Management Studies}
\author{Witek ten Hove}
\date{2024-03-11}

\begin{document}
\maketitle

\renewcommand*\contentsname{Table of contents}
{
\hypersetup{linkcolor=}
\setcounter{tocdepth}{2}
\tableofcontents
}
\bookmarksetup{startatroot}

\chapter*{Preface}\label{preface}
\addcontentsline{toc}{chapter}{Preface}

\markboth{Preface}{Preface}

These are the course notes to the Change Management workshops.

\bookmarksetup{startatroot}

\chapter{Introduction}\label{introduction}

Welcome to F-cluster Change Management.

\begin{longtable}[]{@{}llll@{}}
\toprule\noalign{}
\# & Naam Competentie & Beschrijving Competentie & Beschrijving Niveau
2 \\
\midrule\noalign{}
\endhead
\bottomrule\noalign{}
\endlastfoot
1 & Probleem herkennen en diagnosticeren & De startende BK-professional
is in staat om problemen (cq. vraagstukken) van organisatorische aard te
signaleren, te analyseren en te beoordelen ten behoeve van de
effectiviteit van een organisatie en deze uit te werken tot een
bedrijfskundig vraagstuk. & Kan een bedrijfskundig probleem onderkennen,
mogelijke oorzaken en dwarsverbanden identificeren en beoordelen met
behulp van meerdere bedrijfskundige analysetechnieken. Gaat hierbij op
zoek naar aanvullende relevante en betrouwbare informatiebronnen. \\
2 & Ontwerpen & De startende BK professional is in staat met een
onderzoekende en nieuwsgierige houding op basis van een programma van
eisen, ontwikkelingen en trends en evidence-based practices nieuwe
ideeën te signaleren, te genereren en uit te voeren om de effectiviteit
van een organisatie te verbeteren en bedrijfsprocessen te
(her)ontwerpen. & Genereert en selecteert creatieve ideeën waardoor
nieuwe mogelijkheden of oplossingen worden ontwikkeld en kan deze
toepassen om bedrijfsprocessen en structuren te verbeteren, rekening
houdend met kwaliteitscriteria. \\
3 & Veranderen & De bedrijfskundige professional is in staat om
(complexe) veranderingsprocessen vorm te geven en te (bege)leiden door
verbinding te creëren tussen vakgebieden, structuren, mensen en
processen. Hij hanteert hierbij een op draagvlak gerichte aanpak. & Kan
vanuit een veranderkundige aanpak met bijbehorende rollen een
bedrijfskundige verandering opzetten en begeleiden. Kan diverse passende
veranderstrategieën formuleren en onderbouwen. \\
4 & Evalueren & De startende bedrijfskundige professional is in staat de
effectiviteit van verbeteracties te beoordelen en advies te geven over
verdere ontwikkeling van de organisatie. De evaluatie wordt gebruikt om
het leren van de organisatie te verbeteren en is gericht op de eerder
gemaakte keuzes (diagnose en ontwerp). & Kan voor een
verbeteractie/maatregel/beleid/project beoordelen met toepassing van
bedrijfskundige modellen, bediscussieert daarbij de gestelde normen en
doet aanbevelingen voor structurele verbetering. \\
5 & Onderzoekend vermogen & De startende professional beargumenteert de
keuze voor rol en aanpak om tot oplossing te komen van het
bedrijfskundige vraagstuk. In de aanpak beschikt hij over een
onderzoekende houding van waaruit hij denkt en werkt. Hij past kennis
uit beschikbaar onderzoek toe in de eigen werkpraktijk en ontwerpt en
voert methodisch zelf (kleinschalig) praktijkgericht onderzoek uit. & In
staat zelf onderzoek te doen volgens de werkcyclus van praktijkgericht
onderzoek. \\
6 & Sociaal communicatieve vaardigheden & De startende BK professional
kan zowel in de Nederlandse als Engelse taal ideeën, meningen,
standpunten, en besluiten begrijpelijk en overtuigend overbrengen
afgestemd op de toehoorder en/of in begrijpelijke en correcte taal op
schrift stellen en afstemmen op de lezer. Bewust van de eigen rol in de
interactie met anderen en kan gespreksvaardigheden inzetten om sturing
te geven aan het realiseren van de bedrijfskundige doelen. & Kan
mondeling en schriftelijk communiceren intern op alle niveaus effectief
en in de gangbare bedrijfstaal, veelal in het Nederlands en/of Engels.
Kan een adviesvraag helder krijgen. Weet verschillende gespreks- en
communicatievormen te onderscheiden en gericht toe te passen. \\
7 & Schakelen en verbinden & De startende professional kan waardevolle
(internationale) relaties leggen en allianties aangaan binnen en buiten
de organisatie(keten) voor het verkrijgen van informatie, steun en
medewerking. Hij weet wat er nodig is om relevante verbindingen te
leggen tussen mensen en de verschillende functionele gebieden en niveaus
en kan daar soepel tussen schakelen. & Kan het krachtenveld van in- en
externe actoren van een (tijdelijke) organisatie in kaart brengen. Kan
bestaande relaties beheren en evalueren. \\
8 & Professionaliseren & De startende BK professional is in staat via de
weg van reflectie zich professioneel te blijven ontwikkelen en een
bijdrage te leveren aan de ontwikkeling van de organisatie en de
beroepspraktijk door opgedane ervaringen en feedback te gebruiken voor
persoonlijke groei. & De student reikt nieuwe inzichten aan uit
resultaten uit theorie en praktijk. Kan, op basis van reflectie op
zichzelf en van de omgeving, gestructureerd aan zijn ontwikkeling
werken. \\
9 & Handelen vanuit waarden & De startende bedrijfskundige professional
handelt vanuit een waardenbesef en heeft bij het zoeken naar oplossingen
voor bedrijfskundige vraagstukken oog voor mogelijke consequenties van
(bewuste of onbewuste) keuzes en handelingen op de langere termijn. &
Heeft zicht op de mogelijke consequenties van bedrijfskundig handelen
voor stakeholders, milieu en maatschappij. \\
Overkoepelend & Bedrijfskundig redeneren & De startende professional is
in staat tot onderbouwde argumenten te komen voor een integraal advies
op basis van data, theorie, de bedrijfskundige professionele vakmanschap
en persoonlijke inbreng, zodat besluitvorming hierop plaats kan vinden.
De student heeft zicht op de bedrijfskundige aspecten in relatie tot
elkaar conform het bedrijfskundig model (niveaus/breedte: disciplines).
& De student onderbouwt een bedrijfskundig advies door bedrijfskundige
kennis en vaardigheid toe te passen in het continu proces van
gegevensverzameling en analyse, gericht op de vragen en problemen van
een organisatie en diens stakeholders. \\
\end{longtable}

\bookmarksetup{startatroot}

\chapter{Key principles}\label{key-principles}

\includegraphics{index_files/mediabag/Kotters-8-Steps.jpg}

\begin{longtable}[]{@{}lll@{}}
\toprule\noalign{}
\# & Short Name for Principle & Long Description of Principle \\
\midrule\noalign{}
\endhead
\bottomrule\noalign{}
\endlastfoot
1 & Diagnosis Step \#1 & Gathering facts to understand the need for
change and pre-existing conditions that might affect its
implementation. \\
2 & Diagnosis Step \#2 & Assessing the organization\textquotesingle s
readiness for change, including history with change, stress levels, and
leadership capabilities. \\
3 & Evidence-Based Interventions & Choosing interventions based on
diagnosis, expertise, stakeholder input, and scientific evidence. \\
4 & Change Leadership & Developing leadership skills at all levels to
effectively guide and implement change. \\
5 & Clear Compelling Vision & Creating a distinct and motivating vision
that signals a break from the past. \\
6 & Vision Communication & Communicating the vision across the
organization in an understandable and compelling way. \\
7 & Employee Participation & Encouraging active involvement from
employees to reduce resistance and increase buy-in. \\
8 & Empower and Enable & Empowering employees by providing resources,
training, and removing obstacles. \\
9 & Short-Term Wins & Identifying and celebrating early successes to
build momentum for the change. \\
10 & Monitor and Adjust & Regularly assessing and adjusting the change
process to ensure it is on track. \\
11 & Institutionalize Change & Embedding the change into the
organization\textquotesingle s culture and practices for lasting
impact. \\
\end{longtable}

\bookmarksetup{startatroot}

\chapter{Leadership}\label{leadership}

\bookmarksetup{startatroot}

\chapter{Summary}\label{summary}

In summary, this book has no content whatsoever.

\bookmarksetup{startatroot}

\chapter*{References}\label{references}
\addcontentsline{toc}{chapter}{References}

\markboth{References}{References}

\phantomsection\label{refs}
\begin{CSLReferences}{0}{1}
\end{CSLReferences}



\end{document}
